\documentclass{article}

\usepackage{amsfonts}
\usepackage{amsmath}

\newcommand{\Z}{\mathbb{Z}}

\begin{document}
\title{Application of Discrete Models}
\author{Adam Zlehovszky}
\maketitle

\section{Representation of Integers}

\subsection{Euclidean division}
If $a,b \in \Z$ with $b \ne 0$ then $\exists ! q,r \in \Z$ such that $a=qb+r$ where $0 \le r < |b|$.
This is the \emph{Euclidean division} or \emph{long division} of the \emph{dividend} $a$ with the \emph{divisor} $b$.
The results of the division are the \emph{quotient} $q$ and the \emph{remainder} $r$.
The standard notation for the remainder is $a \bmod b$.

\subsection{Number systems}

Let $1 < b \in \Z$ be the \emph{base} of the \emph{number system}.
For each $0 \le n \in \Z$ there exists a unique $1 \le d \in \Z$ and a unique set of \emph{digits} $0 \le n_1, n_2, \ldots, n_{d-1} < b$ all integers, such that
\[
    n = \sum_{k=0}^{d-1}n_k b^{k}.
\]
If $n = 0$, then $d=1$ and $n_0 = 0$. Otherwise $d = \left \lfloor \log_{b} n \right \rfloor + 1$ and we can extract the digits of $n$ with long division, since
\begin{align*}
    n & = n_{d-1}b^{d-1} + \cdots + n_2 b^2 + n_1 b + n_0 \\
      & = \left( n_{d-1}b^{d-2} + \cdots + n_2 b + n_1 \right)b + n_0 
\end{align*}
where the quotient $n_{d-1}b^{d-2} + \cdots + n_2 b + n_1$ is a $d-1$ digit number and $n_0$ is the extracted digit.

We call $n_0$ the \emph{least significant digit} and $n_{d-1}$ the \emph{most significant digit}.
The storage order of digits is called \emph{little endian} if we start at the least significant digits and move towards the most significant one. Otherwise it is called \emph{big endian}.
\end{document}